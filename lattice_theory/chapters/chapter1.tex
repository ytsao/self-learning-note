\section{Ordered Sets}

An ordering is a binary relation on a set of objects. 

\begin{definition}
    Let $\bm{P}$ be a set. An \text{\textbf{order}} (or \text{\textbf{partial order}}) on $\bm{P}$ is a binary relation $\leq$ on $\bm{P}$ 
    such that for all $x, y, z \in \bm{P}$.
    \begin{itemize}
        \item (reflexivity) $x \leq x$
        \item (antisymmetry) $x \leq y \land y \leq x \implies x = y$
        \item (transitivity) $x \leq y \land y \leq z \implies x \leq z$
    \end{itemize}
\end{definition}

\begin{align*}
    \langle \bm{P}; \leq \rangle &\coloneqq \text{"ordered set" or "partially ordered set", poset.}\\
                            &\coloneqq \text{A set } \bm{P} \text{ equipped with an order relation} \leq. 
\end{align*}

discrete order $\coloneqq$ "=" is an order on any set.\\
guasi-order or pre-order $\coloneqq$  \\
induced order $\coloneqq$ $x, y \in \bm{Q}, x \leq y \iff x \leq y \in \bm{P}, \text{where } \bm{Q} \subseteq \bm{P}.$\\

\begin{definition}
    Let $\bm{P}$ be an ordered set. Then $\bm{P}$ is a \text{\textbf{chain}} if for all $, y \in \bm{P}$, \text{either } $x \leq y \text{ or } y \leq x$.
\end{definition}
Chain can be called, \textit{linearly ordered set}, or \textit{totally ordered set}.\\
\textbf{Antichain} $\coloneqq$ if $x \leq y \text{ in } \bm{P} \text{ only if } x = y$.
Hence, any set $\bm{S}$ may be converted into an antichain $\bar{\bm{S}}$ by giving $\bm{S}$ the discrete order.

\begin{definition}{Order-isomorphisms}
    $\bm{P} \cong \bm{Q}$, if there exists a map $\varphi$ from $\bm{P}$ onto $\bm{Q}$ such that $x \leq y$ in $\bm{P} \iff \varphi(x) \leq \varphi(y)$ in $\bm{Q}$.
    Then $\varphi$ is called an "order-isomorphism".
\end{definition}

\textbf{Note:} order-isomorphism $\Rightarrow$ bijection (one-to-one and onto). Hence, 
\[
    \varphi \colon \bm{P} \rightarrow \bm{Q}, \quad \varphi\inv \colon \bm{Q} \rightarrow \bm{P}
\].

\begin{definition}
    Powerset $\wp(\bm{X})$, consisting of all subsets of x, is ordered by set inclusion:\\
    for $A, B \in \wp(\bm{X})$, we define: $A \leq B \iff A \subseteq B$.
\end{definition}
\textbf{Note:} any subset of $\wp(\bm{X})$ inherits the inclusion order.

\begin{align*}
    Predicate &\coloneqq \text{ A statement taking value } T (\text{true}) \text{ or value } F (\text{false}).\\
    &\coloneqq x \rightarrow \{T, F\}.
\end{align*}
\begin{example}
\[
    p \colon \mathbb{R} \rightarrow \{T, F\}
\]
\[
    p(x) = \begin{cases} T, &\text{if } x \geq 0 \\ 
                        F, &\text{if } x < 0
            \end{cases}
\]
\end{example}
$\mathbb{P}(\bm{X}) \coloneqq$ The set of predicates on $\bm{X}$.\\
Let $p$ and $q$ are predicate. Then we denote by 
\[
    p \stackrel{}{\Rrightarrow} q \iff \{x \in \bm{X} \mid p(x) = T \} \subseteq \{x \in \bm{X} \mid g(x) = T \}.
\]

\begin{definition}{The convering relation}
    Let $\bm{P}$ be an ordered set, $x, y \in \bm{P}$ 
    \[
        \left\{
        \begin{array}{l}
            x \text{ is covered by } y \\
            y \text{ covers } x
        \end{array}
        \right\}.
    \]
    \[
        x \coveredby y \text{ or } y \covering x
    \]
\end{definition}

\textbf{Note:} $x < y \land x \leq z < y \rightarrow x = z$.\\
\textbf{Note:} if $\bm{P}$ is finite, 
\begin{align*}
    x < y \iff &\text{ there exists a finite sequence of covering relations }\\
    &x = x_0 \coveredby x_1 \coveredby x_2 \dots \coveredby x_n = y
\end{align*}

\begin{example}
    In the chain $\mathbb{N}$, we have $m \coveredby n \iff n = m + 1$.
\end{example}
\begin{example}
    In $\mathbb{R}$, there are \textcolor{red}{no pairs} $x, y$ such that $x \coveredby y$.
\end{example}
\begin{example}
    In $\wp(\bm{X})$, we have 
    \[
        \bm{A} \coveredby \bm{B} \iff \bm{B} = \bm{A} \cup \{b\},
    \]
    for same $b \in \bm{X} \setminus \bm{A}$. 
\end{example}

\newpage 

\textbf{Diagrams Rules: (Hasse diagram)}
\begin{enumerate}
    \item To each point $x \in \bm{P}$, associate a point $p(x)$ of the Euclidean plane $\mathbb{R}^2$, depicted by a small circle with centre at $p(x)$.
    \item For each covering pair $x \coveredby y \text{ in } \bm{P}$, take a line segment $\ell(x,y)$ joining the circle at $p(x)$ to the circle at $p(y)$.
    \item Carry out 1. and 2. in such a way that:
        \begin{enumerate}
            \item if $x \coveredby y$, then $p(x)$ is "lower" than $p(y)$.
            \item the circle at $p(z)$ does not intersect the line segment $\ell(x,y)$ if $z \neq x \land z \neq y$.
        \end{enumerate}
\end{enumerate}
By diagram, 
\[
    x < y \iff \text{ there is a sequence of connected line segments moving upwards from } x \text{ to } y.
\]

\begin{lemma}
    Let $\bm{P}$ and $\bm{Q}$ be finite ordered sets and let $\varphi \colon \bm{P} \rightarrow \bm{Q}$ be a bijective map.
    Then the following are equivalent: 
    \begin{enumerate}
        \item $\varphi$ is an order-isomorphism. 
        \item $x < y \text{ in } \bm{P} \iff \varphi(x) < \varphi(y) \text{ in } \bm{Q}$.
        \item $x \coveredby y \text{ in } \bm{P} \iff \varphi(x) \coveredby \varphi(y) \text{ in } \bm{Q}$.
    \end{enumerate}
\end{lemma}

\begin{proposition}
    Two finite ordered sets $\bm{P}$ and $\bm{Q}$ are order-isomorphic iff they can be drawn with identical diagrams.
\end{proposition}

\begin{definition}{The dual of an ordered set}
    Given any ordered set $\bm{P}$, we can form a new ordered set $\bm{P}^{\partial}$ \textcolor{red}{(the dual of $\bm{P}$)} by defining $x \leq y$ 
    to hold in $\bm{P}^{\partial} \iff y \leq x$ holds in $\bm{P}$.
\end{definition}

\textcolor{red}{\textbf{The Duality Principle:} \\ 
Given a statement $\Phi$ about ordered sets which is true in all ordered sets, the dual statement $\Phi^{\partial}$ is also true in all ordered set.
}
\\

\textbf{Note:}
\begin{align*}
    \perp \in \bm{P} &\colon \text{ bottom, if } \perp \leq x, \forall x \in \bm{P}.\\
    \top \in \bm{P} &\colon \text{ top, if } \top \geq x, \forall x \in \bm{P}.
\end{align*}
PS $\perp$ and $\top$ are unique, by duality principle and antisymmetry.

\begin{definition}{Maximal (\textcolor{red}{MaxQ})}
    Let $\bm{P}$ be an ordered set and let $\bm{Q} \subseteq \bm{P}$.
    Then $a \in \bm{Q}$ is a maximal element of $\bm{Q}$ if $a \leq x \land x \in \bm{Q} \implies a = x$.
\end{definition}
\begin{definition}{Minimal (\textcolor{red}{MinQ})}
    Let $\bm{P}$ be an ordered set and let $\bm{Q} \subseteq \bm{P}$.
    Then $a \in \bm{Q}$ is a minimal element of $\bm{Q}$ if $a \geq x \land x \in \bm{Q} \implies a = x$.
\end{definition}
\begin{definition}{Maximum (\textcolor{red}{$\top_Q = maxQ$})}
    If $\bm{Q}$ (with the order inherited from $\bm{P}$) has a top element, $\top_Q$, then $MaxQ = \{\top_Q\}$.
\end{definition}
\begin{definition}{Minimum (\textcolor{red}{$\perp_Q = minQ$})}
    If $\bm{Q}$ (with the order inherited from $\bm{P}$) has a bottom element, $\perp_Q$, then $MinQ = \{\perp_Q\}$.
\end{definition}